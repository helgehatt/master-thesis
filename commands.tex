\usepackage{xparse}

% MATH SHORT-HAND

\NewDocumentCommand{\link}{ s D(){i} D(){j} }{
    \IfBooleanTF{#1}
        {#2 \not\rightarrow #3}
        {#2     \rightarrow #3}
}

\NewDocumentCommand{\sub}{ d() D(){ij} }{ #1_{#2} }

% WRAPPERS

\NewDocumentCommand{\xfig}{ O{!htb} m }{
    \begin{figure}[#1]
        #2
    \end{figure}
}

\NewDocumentCommand{\xtab}{ O{!htb} m }{
    \begin{table}[#1]
        #2
    \end{table}
}

\NewDocumentCommand{\xeq}{ O{!htb} m }{
    \begin{eqfloat}[#1]
        #2
    \end{eqfloat}
}

\NewDocumentCommand{\xwrap}{ O{figure} O{r} m O{0.5\textwidth} }{
    \begin{wrapfloat}{#1}{#2}{#4}
        \renewcommand{\figurename}{Fig}
        \vspace{-\baselineskip}
        #3
        % \vspace{-\baselineskip}
    \end{wrapfloat}
}

\NewDocumentCommand{\xvfill}{ m }{
    \newpage~\vfill#1~\vfill
}

% CONTENT

\NewDocumentCommand{\ximg}{ O{width=\textwidth} D//{figures} m O{} }{
    {\centering
    \includegraphics[#1]{contents/#2/#3}
    \captionof{figure}{#4} 
    \label{#3}\par}
}

\NewDocumentCommand{\xtex}{ O{table} D//{tables} m O{} }{
    {\centering
    \input{contents/#2/#3}
    \caption{#4}
    \label{#3}\par}
}

\NewDocumentCommand{\xanim}{ O{width=\textwidth} D||{12} m O{} }{
    {\centering
    \animategraphics[#1]{#2}{contents/figures/#3}{}{}
    \captionof{animation}{#4}
    \label{#3}\par}
}

% MULTIPLES

\NewDocumentCommand{\xdouble}{ O{0.475} m m }{
    \renewcommand{\figurename}{Fig}
    \floatname{animation}{Anim}
    \begin{minipage}[t]{#1\textwidth}
        #2
    \end{minipage}
    \hfill
    \pgfmathparse{0.95 - #1}
    \begin{minipage}[t]{\pgfmathresult\textwidth}
        #3
    \end{minipage}
}

\NewDocumentCommand{\xtriple}{ O{0.3} O{0.3} m m m }{
    \begin{minipage}[t]{#1\textwidth}
        #1
    \end{minipage}
    \hfill
    \begin{minipage}[t]{#2\textwidth}
        #2
    \end{minipage}
    \hfill
    \pgfmathparse{0.9 - #1 - #2}
    \begin{minipage}[t]{\pgfmathresult\textwidth}
        #3
    \end{minipage}
}

% DRAWING

\NewDocumentCommand{\xmatline}{D(){0} O{red!60} m m}{%
    \draw[line width=1bp, color=#2] ($(#3.north west)+(-#1mm,0mm)$) rectangle ($(#4.south east)+(#1mm,0mm)$);
}

\NewDocumentCommand{\xmatfill}{D(){0} O{red!60} m m}{%
    \draw[draw=none, fill=#2] ($(#3.north west)+(-#1mm,0mm)$) rectangle ($(#4.south east)+(#1mm,0mm)$);
}