\chapter{Data}\label{ch:Data}

Datasets are taken from the \citetitle{snapnets}~\cite{snapnets}, containing hundreds of networks from social networks to collaboration networks as well as various web graphs. The size of the networks ranges from 100 nodes to several millions of nodes, with the largest network having almost two billion edges.
Before testing on real-life networks, synthetic datasets are generated based on the developed models to verify that the models are capable of capturing the underlying processes, as well as giving insight to strengths and weaknesses of the different models. 

\section{Synthetic Data}

    Many communication networks such as social graphs usually comprise several groups of people who frequently interact. Hence, the synthetic data is generated by drawing two-dimensional initial positions for the nodes from multivariate Gaussians with known covariance. 
    Each Gaussian simulates a cluster, where the nodes are connected depending on whether the distance between two nodes is small enough with respect to the bias $\beta$. This is formally defined in \Cref{eq:synth-link} using 0.5 as the classification threshold and where $\sigma(x)$ is the sigmoid function. 
    \begin{equation}\label{eq:synth-link}
        \sub(y) = 
        \begin{cases}
            1 & \text{if } \sigma(\beta - |z_i - z_j|) \geq 0.5 \\
            0 & \text{otherwise}
        \end{cases}
    \end{equation}
    
    The drawn initial positions defines the latent space $Z$ at the first time step. Latent space positions for subsequent time steps are generated by making small changes to the latent space at the previous time step. The change in positions depends entirely on which model the adjustment is supposed to mimic, where a static model assumes no change over time while a diffusion model adds diffusion between each time step. Once latent space positions have been generated for all time steps, links are determined depending on the distance between the nodes, resulting in the target sociomatrix $Y$. To keep the synthetic data simple, only undirected links are considered, as well as limiting the dataset to 30 entities over 100 time steps.
    
    \subsection{Static Data}
    
        Initial positions are sampled from three multivariate Gaussians with means $(\pm2,2)$ as well as $(0,-2)$ and covariance $\Sigma=0.5^2\bm{I}$, as illustrated in \Cref{fig:Synthetic-Initial-0}. Note that these initial positions are used in all synthetic datasets.

        \xfig{Synthetic-Initial-0}[Latent space positions at $t=0$ drawn from three Gaussians with ten samples each]
    
        Which nodes are connected depends entirely on the chosen bias, where \Cref{fig:Synthetic-Bias-Low} illustrates the links using $\beta=0.5$ while \Cref{fig:Synthetic-Bias-High} uses a bias of $\beta=1$. To resemble a more or less sparse network, $\beta=0.5$ is selected as the bias for all datasets, resulting in 64 links at $t=0$ and a sparsity of about 93\%.
        
        \xfigfig{Synthetic-Bias-Low}[Links using a bias of 0.5]{Synthetic-Bias-High}[Links using a bias of 1]
    
        % As previously mentioned, a static model assumes 
        % The static data has no temporal aspect, such that the links do not change over time. 
        
    
    \subsection{Dynamic Data}
    
        The diffusion model considers the observation at time $T$ to be equal to the observation at time $T-1$ with added white noise.
    
        The synthetic diffusion data uses the same initial positions as the static data, with positions for subsequent time steps determined by adding diffusion as $N(0,0.05^2)$. 
        
        \xfig{Synthetic-Dynamic-Links}
        
        \xfig[width=\textwidth]{Synthetic-Dynamic-Autocorrelation}
        
        Impact of diffusion rate
        \xanimanim{Synthetic-Dynamic-Low}[Animation of subsequent time steps adding diffusion with $\sigma=0.01$]{Synthetic-Dynamic-High}[Animation of subsequent time steps adding diffusion with $\sigma=0.25$]
    
    \subsection{Periodic Data}
    
        The autoregressive model is capable of capturing periodic patterns, such that the dynamic data should include periodicity. This is achieved by defining initial positions for the first three time steps, simulating an AR(3) process. 
        
        The first set of initial positions are the same as for the static data, while the second and third set of positions are defined using two Gaussians with means $(\pm2,0)$ and four Gaussians with means $(\pm2,\pm2)$ respectively and a variance of $0.5^2$.
       
        \xfigfig{Synthetic-Initial-1}[Positions at $t=1$]{Synthetic-Initial-2}[Positions at $t=2$]
        
        Similar to the diffusion model, the autoregressive model also includes diffusion, where the same rate of $\sigma=0.05$ is used. Furthermore, the positions for subsequent time steps are calculated according to REF using $\phi=[0, 0, 1]$
        
        
        \xfig{Synthetic-Periodic-Links}
        
        \xfig[width=\textwidth]{Synthetic-Periodic-Autocorrelation}
        
        % \xanim|4|{Synthetic-Periodic-Data}[Animation of the generated periodic data]
        
        \xfigfig{Synthetic-Periodic-Low}[Positions at $t=3$ using $\phi=[0,0,0]$, resulting in white noise with $\sigma=0.01$.]{Synthetic-Periodic-High}[Positions at $t=3$ using $\phi=[1,1,1]$]
        
\section{UC Messaging Data}

    \xfig{UC-Messaging-Data}
    
\section{EU Email Data}

    \xfig{EU-Email-Data}