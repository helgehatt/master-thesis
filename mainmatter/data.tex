\chapter{Data}\label{ch:Data}

Datasets are taken from the \citetitle{snapnets}~\cite{snapnets}, containing hundreds of networks from social networks to collaboration networks as well as various web graphs. The size of the networks ranges from 100 nodes to several millions of nodes, with the largest network having almost two billion edges.
Before testing on real-life networks, synthetic datasets are generated based on the developed models to verify that the models are capable of capturing the underlying processes, as well as giving insight to strengths and weaknesses of the different models. 

\section{Synthetic Data}

    Many communication networks such as social graphs usually comprise several groups of people who frequently interact. Hence, the synthetic data is generated by drawing two-dimensional initial positions for the nodes from multivariate Gaussians with known covariance. 
    Each Gaussian simulates a cluster, where the nodes are connected depending on whether the distance between two nodes is small enough with respect to the bias $\beta$. This is formally defined in \Cref{eq:synth-link} where $\sigma(x)$ is the sigmoid function (cf. eq. \ref{eq:lsm-link-sigmoid}) and $\alpha$ is drawn from a uniform distribution to simulate a link with probability given by the logistic regression.
    \begin{equation}\label{eq:synth-link}
        \sub(y) = 
        \begin{cases}
            1 & \text{if } \sigma(\beta - |z_i - z_j|) \geq \alpha \qquad \alpha\sim U(0,1) \\
            0 & \text{otherwise}
        \end{cases}
    \end{equation}
    
    The drawn initial positions forms the latent space $Z$ at the first time step. Latent space positions for subsequent time steps are generated by making small changes to the latent space at the previous time step. The change in positions depends entirely on which model the adjustment is supposed to mimic, where a static model assumes no change over time while a diffusion model adds diffusion between each time step. Once latent space positions have been generated for all time steps, links are drawn according to the link probability, resulting in the target sociomatrix $Y$. To keep the synthetic data simple, only undirected links are considered, as well as limiting the dataset to 30 entities over 100 time steps.
    
    \subsection{Static Data}
    
        Initial positions are sampled from three multivariate Gaussians with means $(\pm2,2)$ as well as $(0,-2)$ and covariance $\Sigma=0.5^2\bm{I}$, as illustrated in \Cref{Data/Static/Initial0}. 
        \xfig{\ximg[width=0.7\textwidth]{Data/Static/Initial0}[Latent space positions at $t=0$ drawn from three Gaussians with ten samples each]}
        The number of connected nodes depends on the chosen bias, which is used to skew the link probabilities in either direction. \Cref{Data/Static/BiasLow} illustrates the links using a bias of $\beta=-1$ while \Cref{Data/Static/BiasHigh} uses a bias of $\beta=1$. The difference is 47 links opposed to 160 using the positive bias, including several links across clusters. 
        \xdouble{\ximg{Data/Static/BiasLow}[Links using a bias of -1]}{\ximg{Data/Static/BiasHigh}[Links using a bias of 1]}
        Real-life networks are usually quite sparse, such that the negative bias of $\beta=-1$ is used in all synthetic datasets, resulting in the number of links shown in \Cref{Data/Static/Links}. Even though the latent space positions remain constant over time, the positions are only used to find the probabilities of links occurring, such that the number of links vary over time, while the average number of links is independent of time.
        \xdouble{\ximg{Data/Static/Links}[Number of links as a function of time]}{\xanim{Data/Static/Data}[Animation of the static data including links between nodes]}
        \Cref{Data/Static/Data} shows an animation of the synthetic static dataset for all 100 time steps (click on the image in Adobe Reader to play). The image sequence shows that the latent space positions do not change over time, while the links between nodes appear at random with higher frequency for close proximity nodes.
    
    \subsection{Dynamic Data}
    
        The dynamic data is based on the same initial positions as the static data, however assumes that the latent positions for subsequent time steps are equal to the previous with constant diffusion added. This simulates the effect of observations drifting apart, each in a particular direction with constant speed. The diffusion is sampled from a normal distribution with zero mean and $\sigma_\epsilon^2$ variance, termed the diffusion rate.
        
        Impact of diffusion rate
        \xdouble{\xanim{Data/Dynamic/DiffusionLow}[Animation of subsequent time steps adding diffusion with $\sigma_\epsilon=0.01$]}{\xanim{Data/Dynamic/DiffusionHigh}[Animation of subsequent time steps adding diffusion with $\sigma_\epsilon=0.25$]}
        
        \xdouble{\ximg{Data/Dynamic/Links}}{\xanim{Data/Dynamic/Data}}
    
    \subsection{Periodic Data}
    
        The autoregressive model is capable of capturing periodic patterns, such that the dynamic data should include periodicity. This is achieved by defining initial positions for the first three time steps, simulating an AR(3) process. 
        
        The first set of initial positions are the same as for the static data, while the second and third set of positions are defined using two Gaussians with means $(\pm2,0)$ and four Gaussians with means $(\pm2,\pm2)$ respectively and a variance of $0.01^2$.
       
        \xdouble{\ximg{Data/Periodic/Initial1}[Positions at $t=1$]}{\ximg{Data/Periodic/Initial2}[Positions at $t=2$]}
        
        Similar to the diffusion model, the autoregressive model also includes diffusion, where the same rate of $\sigma=0.01$ is used. Furthermore, the positions for subsequent time steps are calculated according to REF using $\phi=[0, 0, 1]$
        
        \xdouble{\ximg{Data/Periodic/ARNone}[Positions at $t=3$ using $\phi=[0,0,0]$, resulting in white noise with $\sigma=0.01$.]}{\ximg{Data/Periodic/ARAll}[Positions at $t=3$ using $\phi=[1,1,1]$]}
        
        \xdouble{\ximg{Data/Periodic/Links}}{\xanim|3|{Data/Periodic/Data}[Animation of the generated periodic data]}
        
    \subsection{Velocity Data}
    
        $\sigma=0.001$
        
\section{UC Messaging Data}

    \xfig{UC-Messaging-Data}
    
\section{EU Email Data}

    \xfig{EU-Email-Data}