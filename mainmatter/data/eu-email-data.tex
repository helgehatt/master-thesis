\section{EU Email Data}

The EU email network was generated using email data from a large European research institution, containing anonymized information of all incoming and outgoing email between the members. The dataset comprises 986 nodes and 332,334 temporal edges, spanning a total of 803 days. The network is directed, where each edge $(u,v,ts)$ corresponds to person $u$ sending an email to person $v$ at time $ts$. Sending the same email to several people results in separate edges for each recipient. The timestamps $ts$ are in seconds, starting from zero, such that the network's sociomatrix $\tY$ is created by aggregating seconds into days. Doing so results in the number of links over time depicted in \Cref{Data/EUEmail/Links}.

\xfig{\ximg[width=0.7\textwidth]{Data/EUEmail/Links}[EU Email Links]}

The last third of the data is incomplete, however 500 time steps are more than enough to perform a temporal analysis. 
The dataset exhibits a large amount of periodic behavior, which most likely relates to weekdays and weekends, having a lot of activity over five days followed by a two days pause. The periods of little or no activity (around $t=60$ and $t=420$) occur with about a one year gap, while the longer period with slowly declining and inclining activity is in between (around $t=300$). This change in activity is certainly not coincidental, and most likely relates to Christmas and summer vacations, respectively.

To limit the amount of time spent on model optimization, the timespan is constrained to 200 time steps, giving rise to the two segments depicted in \Cref{Data/EUEmail/Links0-200,Data/EUEmail/Links200-400}.
\xfig{\xdouble
    {\ximg{Data/EUEmail/Links0-200}[Links for {$t=[0,200)$}]}
    {\ximg{Data/EUEmail/Links200-400}[Links for {$t=[200,400)$}]}
}
The dynamic models are not expected to predict the yearly seasonal variations, as that would require a lot more than 200 days of training data. Considering that the difference between the two segmented networks is simply unpredictable holidays, only the first segment is examined.