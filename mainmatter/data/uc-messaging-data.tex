\section{UC Messaging Data}

The UC messaging dataset is comprised of private messages sent on an online social network at the University of California, Irvine. The network comprises a total of 1899 nodes with 59,835 temporal edges over 193 days. Similar to the EU email data, the network is directed, where each edge $(u,v,ts)$ corresponds to user $u$ sending a private message to user $v$ at time $ts$. Aggregating the timestamps to days results in the number of links over time shown in \Cref{Data/UCMessaging/Links}.

Unlike the EU email network, the UC messaging network has significantly less structure, with no distinct periodic patterns. However, considering that the links are from an online messaging platform, the behavior is not entirely random. The initial increase in activity reflects how the platform becomes very popular after its release, exchanging a total of 1,200 messages on a single day at its peak. Unfortunately, popularity can be quite temporary, which is expressed by the diminishing activity resulting in a small group of core users after two months.

\xfig{\xdouble
    {\ximg{Data/UCMessaging/Links}[UC Messaging Links]}
    {\ximg{Data/UCMessaging/Links75-195}[Links for {$t=[75,195)$}]}
}

The irregular pattern over the first few months will likely disrupt the predictions of an autoregressive model, considering that one-time shocks affect values infinitely far into the future. This is due to the nature of the AR process, where each subsequent prediction is a multiplicative result of all previous values. Hence, a more fitting dataset is shown in \Cref{Data/UCMessaging/Links75-195}, using only the last 120 time steps.

