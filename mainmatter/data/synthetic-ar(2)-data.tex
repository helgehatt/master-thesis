\section{Synthetic AR(2) Data}

The autoregressive model is capable of modeling more than just diffusion and periodicity, exploiting the dependence between previous time steps. By using the predictions for the last two time steps, features such as velocity, acceleration and curvature can be modeled. This is illustrated using new initial positions, shown in \Cref{Data/Velocity/Initial3} for the first time step. 
\xfig{\ximg[width=0.7\textwidth]{Data/Velocity/Initial3}[Latent space positions at $t_0$ for the synthetic AR(2) data]}
The displacement of the first set of initial positions to the second set of initial positions is found using $\phi=[1,-1]$, such that $Z_2 = Z_1 - Z_0 + \epsilon$. This idea can be extended to characterize particular patterns as shown in the following datasets. To focus solely on these aspects, the diffusion rate $\sigma_\epsilon$ is set to zero.
    
\subsection{Velocity Data}
    
    Velocity is captured as constant change in latent positions between time steps in a specific direction. This is generated by defining a second set of initial positions with all observations moved slightly closer to the middle, specifically by $\pm0.05$ in both dimensions. As previously shown, this change is found using $\phi=[1,-1]$, hence velocity is modeled using $\phi=[2,-1]$ to add the change to the most recent set of observations. This corresponds to the autoregressive process shown in \Cref{eq:data-velocity}, omitting $\epsilon$.
    \begin{equation}\label{eq:data-velocity}
        Z_t = 2 Z_{t-1} - Z_{t-2} = Z_{t-1} + (Z_{t-1} - Z_{t-2})
    \end{equation}
    At each time step, the initial change of $\pm0.05$ is added, such that the adjustment remains constant over time, and in the same direction. The number of links over time is shown in \Cref{Data/Velocity/Links} with an animation of the velocity dataset in \Cref{Data/Velocity/Data}.
    \xfig{\xdirow
        {\ximg{Data/Velocity/Links}[Number of links as a function of time]}
        {\xanim{Data/Velocity/Data}[Animation of the velocity data]}
    }
    The number of links increases significantly as the two clusters intertwine, having minimal distance between observations, and decreases as the nodes are separated once again.
    
\subsection{Acceleration Data}

    Acceleration is quite similar to velocity, the only difference being that the change between time steps increases or decreases over time. This is achieved by modifying the values of $\phi$ slightly, such that $\phi=[2+\theta,-1-\theta]$ where the sign of $\theta$ indicates acceleration or deceleration and the magnitude of $\theta$ decides the rate of change. This effect is illustrated in \Cref{DataAccelerationPhi} having an initial change of 1.0 between the first two time steps and generating subsequent values using $\theta=-0.1$ (deceleration), $\theta=0.0$ (constant) and $\theta=0.1$ (acceleration).
    \xtab{\xtex{DataAccelerationPhi}[Example of how acceleration affects changes over time using different values of $\theta$]}
    
    The change between time steps increases or decreases exponentially, such that continuing in this fashion for 100 time steps will quickly result in changes either too large or too small. Hence, the deceleration data shown in \Cref{Data/Acceleration/DataNegative}, is generated using significantly smaller magnitudes, having an initial change of 0.1 between the first two time steps and a minimal $\theta$ of -0.03.
    \xfig{\xdirow
        {\ximg{Data/Acceleration/LinksNegative}[Number of links as a function of time]}
        {\xanim{Data/Acceleration/DataNegative}[Animation showing deceleration]}
    }
    The deceleration data is included for illustrative purposes, whereas the dataset used to assess model behavior is shown in \Cref{Data/Acceleration/DataPositive}, using an initial change of only 0.01 and a positive $\theta$ of $0.03$.
    \xfig{\xdirow
        {\ximg{Data/Acceleration/LinksPositive}[Number of links as a function of time]}
        {\xanim{Data/Acceleration/DataPositive}[Animation of the acceleration data]}
    }

\subsection{Curvature Data}

    Having a notion of acceleration, curvature is generated by applying different changes to each of the $k$ latent dimensions. This is done by combining the results of the acceleration and deceleration data, starting with a larger initial change in the $x$-axis and decelerating, while having a smaller initial change in the $y$-axis and accelerating. Over time, the change across the $y$-axis will catch up with the change in $x$-coordinates, thus shifting the direction of the observations. This phenomena is depicted in \Cref{Data/Curvature/Data}, defining the curvature dataset.
    \xfig{\xdirow
        {\ximg{Data/Curvature/Links}[Number of links as a function of time]}
        {\xanim{Data/Curvature/Data}[Animation of the curvature data]}
    }
    Notice the similarity in the number of links over time between acceleration (\cref{Data/Acceleration/LinksPositive}) and curvature (\cref{Data/Curvature/Links}). While the dimensionality of $\phi$ can be used to define complex movements over time, an AR(2) process can just as easily capture the curvature data using a one-dimensional $\phi$, as the model is only concerned with the interactions between entities and not with the true underlying circumstances in which they occur.